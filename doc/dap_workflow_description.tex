\section{DAP workflow description}
\label{dap_sec:dap_workflow}

The DAP is divided into parts and blocks, the first part (blocks 1-6)
will deliver the High Level Data products (see Figure
\ref{dap_fig:dap_workflow_1}), the second part (blocks 7-9) will
deliver the Model dependent data products (see Figure
\ref{dap_fig:dap_workflow_2}). 

Each block is responsable for a set of operations, such as reading
data, fitting the input spectra. The procedure that executes an
operation is called ``main module''. The main modules comunicates
between them and between the various blocks via other procedures,
which are called ``interface module''. The main and interfaces modules
can call other modules, which are called ``unitily modules''. The
interface module is therefore responsable to get input files from a
previous interface, convert them in a readable format readable for the
main module, collect the outputs of the main module and convert them
in a form readable for the next interface. In this way, it is
relatively easy to change the software responsible for a specific task
(i.e. replacing the main module responsible for the spectral fitting)
by chaning its interface.  Table \ref{dap_tab:modules} list a summary
of the blocks, interfaces, main modules, and utilities.


\begin{table}
\begin{scriptsize}
\caption{Summary of blocks, interfaces and main modules distribution.}
\begin{tabular}{l |l |l | l}
BLOCK         & Interface modules               & Main  modules    & Utility modules   \\
\hline
\hline
%{\bf Block 1} &                                 &                                \\  
              &                                 &                               &  \\  
{\bf Block 1} & mdap\_read\_datacube            & mdap\_calculate\_spectrum\_sn  &    \\
\hline
%{\bf Block 2} &                                 &                                \\  
              &                                 &                             &\\  
{\bf Block 2} & mdap\_spatial\_binning          & mdap\_voronoi\_2d\_binning  & mdap\_calibrate\_sn   \\
 \hline
%{\bf Block 3} &                                 &                                \\  
              &                                 &                            &    \\  
{\bf Block 3} & mdap\_log\_rebin                & mdap\_do\_log\_rebin        & mdap\_convol\_sigma   \\
\hline
%{\bf Block 4} &                                 &                                \\  
              &                                 &                                 &   \\  
{\bf Block 4} & mdap\_spectral\_fitting         & mdap\_calculate\_spectrum\_sn   & mdap\_range.pro \\
              & mdap\_create\_starting\_guesses & mdap\_bvls(\_external)          & mdap\_stc.pro \\
              &                                 & mdap\_dust\_calzetti            & mdap\_sgn.pro \\
              &                                 & mdap\_get\_losvd                & mdap\_round\_str.pro \\
              &                                 & mdap\_gandalf\_wrapf            & mdap\_interpolate\_2dmaps \\
              &                                 & mdap\_gandalf                   & mdap\_ppxf\_convol\_fft \\
              &                                 & mdap\_ppxf                      &  \\
              &                                 &{\tt mpfit package}              &  \\ 
\hline
%{\bf Block 5} &                                 &                                         \\  
              &                                 &                                         & \\  
{\bf Block 5} & mdap\_measure\_indices    & mdap\_read\_indices\_definitions  & mdap\_round\_str.pro \\
              &                                 & mdap\_do\_measure\_index               &  mdap\_convol\_sigma  \\
\hline
%{\bf Block 6} &                                 &                                         \\  
              &                                 &                                        &   \\  
{\bf Block 6} & mdap\_spatial\_radial\_binning  & mdap\_calculate\_spectrum\_sn      &  mdap\_convol\_sigma   \\
              & mdap\_spectral\_fitting       &   mdap\_do\_measure\_index           & mdap\_get\_losvd     \\ 
              & mdap\_measure\_indices          &   mdap\_dust\_calzetti             & mdap\_range.pro  \\
              & mdap\_do\_kinemetry.pro         &   {\tt mpfit package}              &  mdap\_stc.pro \\
              &                                 &   mdap\_sgandalf                   & mdap\_sgn.pro  \\
              &                                 &   mdap\_gandalf\_wrapf             &  mdap\_interpolate\_2dmaps \\
              &                                 &   mdap\_gandalf                    & mdap\_read\_indices\_definitions\\
              &                                 &   mdap\_ppxf                       & mdap\_ppxf\_convol\_fft\\
              &                                 &   mdap\_bvls(\_external)           &  \\
              &                                 &   mdap\_kinemetry.tex              &\\
              &                                 &   mdap\_do\_k\_rprofiles.pro       & \\
\hline
              &                                 &  &  \\  
{\bf Block 7} &                                 &                                &  \\
\hline
              &                                 &  &  \\  
{\bf Block 8} &                                 &                                &  \\
\hline
              &                                 &  &  \\  
{\bf Block 9} &                                 &                                &  \\
\hline
\hline
\end{tabular}
\end{scriptsize}
\label{dap_tab:modules}
\end{table}

The input datacube is read in block 1 and information (vectors with
galaxy spectra, errors, wavelenght, and spatial information) are
passed to the block 2 for spatial binning. Three spatial binnings are
foreseen, depending on the scientific requirements. The current DAP
version uses the Voronoi binning scheme, (as implemented in IDL by
Cappellari \& Copin 2003), as main module for the spatial binning
task.

Binned spectra are passed to block 3 for logarithmic resampling of the
galaxy spectra (and error) and the stellar templates. Stellar
templates are also broadened to match the instrumental set up. For
this, the instrumental $LSF(\lambda)$ is required as input. Three sets
of log-sampled galaxy spectra are produced, one set for each spatial
binning.

The log-sampled spectra are then passed to block 4 for spectroscopic
measurements. Three fits are performed in block 4, one for each
spatial sampling defined in block 2. Before each fit, the log-sampled
Galaxy spectra are corrected for Milky Way extinction(input
parameter). Results from the first execution will be used to constrain
the fit of the second fit, and soforth. The current version uese: i)
the pPPXF.pro and gandalf.pro (Cappellari \& Emsellem 2004; Sarzi et
al. 2006); and ii) Calzetti et al. (2000) formulas for reddening
correction as main modules in block 4 (See Section
\ref{dap_sec:mdap_spectral_fitting} for further details). Fitting
procedures have been modified to allow the use of external fortran
routine, instrumental velocity dispersion variable with wavelength,
and fitting parameter boundaries from user input.

The output of block 4 are the kinematic
parameters of stars and gas, emission lines fluxes and equivalent
width, reddening, the weights of the stellar templates used in the fit
(for stellar population measurements), and rest-framed galaxy spectra.

Input galaxy spectra (with best-fit emission lines removed) will be
passed to block 5 for measurement of the line strenght. The current
design foresees that absorption line strength will be measured only
onto spectra associated to the first spatial binning (i.e. those with
the higher $S/N$. The current version uses the absoprtion line indices
as defined by Worthey et al. (1994).

Rest-framed spectra, Kinematic measurements, emission line fluxes,
absorption and emission line equivalent widths are then passed to
block 6 for the extraction of the radial profiles of the measured
quantities, and kinemetric analysis.

\subsection{DAP inputs and outputs}
\label{dap_sec:dap_inputs_outputs}

The Data Analysis Pipeline consists in a IDL procedure,
manga\_dap.pro, and a set of ``interfaces'' modules, ``main'' modules,
and ``utilities'', which are described in this document.

To run the pipeline, the following files are needed.
\begin{itemize}

\item total\_filelist.dat. This is the file that specifies the
  galaxies to analyse and their properties. The file must contain 5
  columns. The first column indicates the names of the $N$ datacubes
  (stored in .fits file, see Section
  \ref{dap_sec:mdap_read_datacube}), the second column provides an
  estimate of the galaxy redshift (in km/sec), the third column
  provides an estimate of the stellar velocity dispersion (in km/sec),
  the 4th column provides the mean galaxy ellipticity, and the 5th
  column provides the mean galaxy position angle.

\item all the $N$ files (datacubes or rss) listed in the file total\_filelist.dat.

\item a set of stellar templates, in fits file format.

%\item emission\_lines\_setup\_with\_Balmer\_decrement. A file containing the definitions of the emission lines to include in the fit (see Section \ref{dap_sec:mdap_spectral_fittig}).

%\item absorption\_line\_indices\_definition.dat. A file containing the definitions of the absorption line indices to measure (see Section \ref{dap_sec:mdap_measure_indices}). 
\item A file containing the definitions of the emission lines to include in the fit (see Section \ref{dap_sec:mdap_spectral_fittig}).

\item A file containing the definitions of the absorption line indices to measure (see Section \ref{dap_sec:mdap_measure_indices}). 

\item A configuration file, that contains all the parameters needed in the analysis.
\end{itemize}

The DAP is executed by the following IDL command line 

\[
{\tt IDL> manga\_dap,i,configuration\_file}
\]

where {\tt i} is the index number of the $i$-th entry in the
total\_filelist.dat, column 1 to analyse, $i=0,
N-1$. configuration\_file is a string specifying the name of the
configuration file. For a full description of the configuration file,
see Section \ref{dap_sec:configuration}.


As output, the DAP returns a multilayer fits file with all the
measured quantities (({\tt <datacube\_name> \_high\_level.fits}), an
idl session with all the session variables stored ({\tt
  <datacube\_name>\_mdap\_ session.idl}).  and a log file
(<datacube\_name>\_mdap.log).

The content of the {\tt <datacube\_name>\_high\_level.fits} output
file is described in Table \ref{dap_tab:output}.


\begin{center}
\begin{longtable}{p{0.5cm}|p{3.5cm}| p{10.1cm}}
\caption{Extension description of the DAP output fits file.} \label{dap_tab:output} \\
\hline
{\bf Ext} &  {\bf Name} &{\bf Description} \\
\hline
\endfirsthead
\hline
{\bf Ext} &  {\bf Name} &{\bf Description} \\
\hline
\endhead
\hline
\endlastfoot
\hline
%{\bf Ext} &  {\bf Name} &{\bf Description} \\
0 & signal              & Mean signal per pixel, produced by mdap\_read\_datacube.pro (Section \ref{dap_sec:mdap_read_datacube}). \\
1 & noise               & Mean noise per pixel, produced by mdap\_read\_datacube.pro (Section \ref{dap_sec:mdap_read_datacube}).\\
2 & binning map 1       & Location and geometry of the spatial bins of the first binning scheme.\\ 
3 & binning 1 data      & Measurements performed on the first binning scheme (absorption line indices).\\ 
4 & binning map 2       & Location and geometry of the spatial bins of the second binning scheme.\\ 
5 & binning 2 data      & Measurements performed on the second binning scheme (stellar kinematics). \\ 
6 & binning map 3       & Location and geometry of the spatial bins of the third binning scheme.\\ 
7 & binning 3 lines      & Measurements performed on the third binning scheme: emission line properties (reddening, intensity, flux, and equivalent width).\\ 
8 & binning 3 kinematics      & Measurements performed on the third binning scheme: mean kinematics, and kinematics of individual emission lines.\\ 
9 & binning map radial  & Location and geometry of the radial binning scheme.\\ 
10 & binning radial data & Measurements performed on the radial binning scheme (absorption line indices and stellar velocity dispersion).  \\ 
11 & Stars rotation & Kinemetric measurements on the stellar kinematics (kinematic position angle, rotation curve, inflows/outflows).  \\ 
12 & Gas rotation   & Kinemetric measurements on the gas kinematics (kinematic position angle, rotation curve, inflows/outflows).  \\ 
13 & profiles       & Stellar radial profiles of $\lambda(R)$, $V/\sigma(R)$, and $\sigma(R)$.\\
\hline
\end{longtable}
\end{center}

\subsubsection{Extension 3: outputs related to the first binning
  scheme} 
\label{}
The quantities measured and stored using the first binning scheme are
the following:

\begin{itemize}

\item Column 1. X. The X coordinates (in arcsec) of the centers of the spatial bins. The center of the galaxy has coordinates (0,0).
\item Column 2. Y. The Y coordinates (in arcsec) of the centers of the spatial bins. The center of the galaxy has coordinates (0,0).
\item Column 3. AREA\_BIN. Area in arcsec$^2$ of the spatial bin.
\item Column 4. STON. Estimate of the S/N of the spectrum in the
  spatial bin. The signal is defined as the median of the best fit
  model, the noise as the robust\_sigma of the residuals (observed
  spectrum - best fit).
\item Column 5. NELEMENTS. Number of spectra coadded in the spatial bin.
\item Columns 6-end. Equivalent width of the absorption line indices
  and their errors (Units \AA\ or magnitudes, depending on the index
  definition. The measured indices (and their names) are defined in a
  user-provided file. The name of this file is specified in the
  configuration file (see Section \ref{dap_sec:configuration}).
\end{itemize}

\subsubsection{Extension 5: outputs related to the second binning
  scheme} 
\label{}
The quantities measured and stored using the second binning scheme are
the following:

\begin{itemize}

\item Column 1. X. The X coordinates (in arcsec) of the centers of the spatial bins. The center of the field of view has coordinates (0,0).
\item Column 2. Y. The Y coordinates (in arcsec) of the centers of the spatial bins. The center of the field of view has coordinates (0,0).
\item Column 3. AREA\_BIN. Area in arcsec$^2$ of the spatial bin.
\item Column 4. STON. Estimate of the S/N of the spectrum in the
  spatial bin. The signal is defined as the median of the best fit
  model, the noise as the robust\_sigma of the residuals (observed
  spectrum - best fit).
\item Column 5. NELEMENTS. Number of spectra coadded in the spatial bin.
\item Columns 6-7. VEL and VEL\_ERR. Measured stellar velocity and its error in km/sec.
\item Columns 8-9. DISP and DISP\_ERR. Measured stellar velocity dispersion and its error in km/sec.
\item Columns 10-11. H3 and H3\_ERR. Measured Gauss-Hermite moment of the stellar velocity distribution h3 and its error (Warning: the maximum range allowed in the
  fitting procedure is : $-0.4 < {\rm H3} < 0.4$).
\item Columns 12-13. H4 and H4\_ERR. Measured Gauss-Hermite moment of the stellar velocity distribution h4 and its error (Warning: the maximum range allowed in the
  fitting procedure is : $-0.4 < {\rm H4} < 0.4$).
\item Column 14. CHI2. Chi-squared from the fit.
\end{itemize}

\subsubsection{Extension 7: outputs related to the third binning
  scheme. Emission line properties} 


The quantities measured and stored using the third binning scheme are
the following:

\begin{itemize}

\item Column 1. X. The X coordinates (in arcsec) of the centers of the spatial bins. The center of the galaxy has coordinates (0,0).
\item Column 2. Y. The Y coordinates (in arcsec) of the centers of the spatial bins. The center of the galaxy has coordinates (0,0).
\item Column 3. AREA\_BIN. Area in arcsec$^2$ of the spatial bin.
\item Column 4. STON. Estimate of the S/N of the spectrum in the
  spatial bin. The signal is defined as the median of the best fit
  model, the noise as the robust\_sigma of the residuals (observed
  spectrum - best fit).
\item Column 5. NELEMENTS. Number of spectra coadded in the spatial bin.
%\item Columns 6-7. VEL and VEL\_ERR. Flux-weighted mean velocity of the emission lines and its error in km/sec.
%\item Columns 8-9. DISP and DISP\_ERR. Flux-weighted mean velocity dispersion of the emission lines and its error in km/sec.
\item Columns 6-7. E(B-V) color excess for the stellar component and its error
\item Columns 8-9. E(B-V) color excess for the ionized gas stellar component and its error
\item Column 10.  CHI2. Chi-squared from the fit.
\item Columns 11-end. Intensities, Intensities errors, Flux, flux error, Equivalent widths, and
  equivalent width errors of the emission lines.  The measured
  emission lines (and their names) are defined in a user-provided
  file. The name of this file is specified in the configuration file
  (see Section \ref{dap_sec:configuration}). Warning: intensities and fluxes are per arcsec$^2$, i.e. the numbers measured in the spectrum are divided by the area of the spatial bin (column 3).
\end{itemize}

\subsubsection{Extension 8: outputs related to the third binning
  scheme. Kinematics} 

The quantities measured and stored using the third binning scheme are
the following:

\begin{itemize}
\item Column 1. X. As in extension 7.
\item Column 2. Y. As in extension 7.
\item Columns 3-4. MEAN\_VEL and ERR\_MEAN\_VEL. Flux-weighted mean velocity of the emission lines and its error in km/sec.
\item Columns 5-6. MEAN\_DISP and ERR\_MEAN\_DISP. Flux-weighted mean velocity dispersion of the emission lines and its error in km/sec.
\item Columns 7-end. Velocities, velocity errors, velocity dispersions and velocity dispersino errorrs measured for the individual emission lines.
\end{itemize}



\subsubsection{Extension 10: outputs related to the radial binning
  scheme} 


The quantities measured and stored using the radial binning scheme are
the following:

\begin{itemize}

\item Column 1. AMAJ. Lenght of the semi-major axis (in arcsecs) describing the elliptical bin. AMAJ=0 is the center of the field of view.
\item Column 2. AMAJ\_LO. Lower limit boundary (in arcsec) of elliptical bin.
\item Column 3. AMAJ\_UP. Upper limit boundary (in arcsec) of elliptical bin.
\item Column 4. STON. Estimate of the S/N of the spectrum in the
  spatial bin. The signal is defined as the median of the best fit
  model, the noise as the robust\_sigma of the residuals (observed
  spectrum - best fit).
\item Columns 5-6.  DISP and DISP\_ERR. Measured stellar velocity dispersion and its error in km/sec.
\item Column 7.  CHI2. Chi-squared from the fit.
\item Columns 8-end. Equivalent width of the absorption line indices
  and their errors (Units \AA\ or magnitudes, depending on the index
  definition. The measured indices (and their names) are defined in a
  user-provided file. The name of this file is specified in the
  configuration file (see Section \ref{dap_sec:configuration}).
\end{itemize}

\subsubsection{Extension 11 (12): outputs related to the kinemetric analysis of stellar (gas) kinematics}
The quantities measured and stored in extension 10 (11) are:
\begin{itemize}
\item Column 1. Elliptical semimajor axis (computed fixing the systemic velocity and the kinematic center, position angle, and axial ratio).
\item Column 2. Mean kinematic position angle.
\item Column 3. Standard deviation of the kinematic position angles at each semimajor axis.
\item Column 4. Mean kinematic axial ratio.
\item Column 5. Standard deviation of the kinematic axial ratios at each semimajor axis.
\item Column 6. Systemic velocity.
\item Column 7. Standard deviation of the systemic velocities at each semimajor axis.
\item Column 8. Rotation velocity measured at each semimajor axis.
\item Column 9. Error on the rotation velocity measured at each semimajor axis.
\item Column 10. Expansion velocity measured at each semimajor axis.
\item Column 11. Error on the expansion velocity measured at each semimajor axis.
%\item Column 12. X coordiante of the assumed center of rotation (0,0 is the center of the field of view).
%\item Column 13. Y coordiante of the assumed center of rotation (0,0 is the center of the field of view).

\end{itemize}



See Section \ref{dap_sec:mdap_do_kinemetry} for more information on how these quantities are computed.

