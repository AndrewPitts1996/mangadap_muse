\subsection{{\tt mdap\_spectral\_fitting.pro}}
\label{dap_sec:mdap_spectral_fitting}

This module is responsable for i) arranging the inputs from previous
modules/blocks and feed them into the main module that perform the fit
(mdap\_gandalf\_wrap.pro, see Section
\ref{dap_sec:mdap_gandalf_wrap.pro}) , and ii) re-arrange the outputs
of the fits for further processing.

The list of input/output parameters of mdap\_spectral\_fitting.pro is
given in Table \ref{dap_tab:mdap_spectral_fitting}.



%\begin{table}
\begin{center}
\begin{longtable}{p{2.7cm}| p{11.1cm}}
\caption{Inputs and outputs parameters of mdap\_spectral\_fitting.pro} \label{dap_tab:mdap_spectral_fitting} \\
\hline
\endfirsthead

\hline
\endhead

\hline
\endlastfoot

%\begin{tabular}{p{2.7cm}| p{2.5cm} |p{9cm}}
\hline
{\bf  INPUTS} &  \\
\hline
 galaxy &     [MxN  array]. It contains the N galaxy spectra to fit, logarithmically 
              sampled (natural log). Units: 1e-17 erg/s/cm$^2$/Angstrom. Spectra are defined of the
              M elements array loglam\_gal (see below).\\
%
  noise &     [MxN  array]. It contains the N error vectors for the
              N galaxy spectra. Same units as galaxy. Error vectors are defined of the
              M elements array loglam\_gal (see below).\\
%
  loglam\_gal& [M array]. It contains the log wavelength values where
             galaxy and noise spectra are defined. The vector must have a constant 
             log(angstrom) sampling. \\
%
  templates & [MM x NN array].  It contains the NN stellar template
            spectra, logarithmically sampled at the same \kms/pixel as the
            galaxy spectra. Same units as galaxy, except an arbitrary
            multiplicative factor.\\
%
 loglam\_templates & [MM dblarray]. It contains the log wavelength values where
            templates are sampled. It must have a constant log(angstrom) sampling.  \\
%
 velscale & [float].  Defines the (uniform) sampling of the input spectra, in \kms/pixel.\\
%;
\hline
%
{\bf OPTIONAL INPUTS } &  \\
\hline
 extra\_inputs & [string array] It contains other inputs
               that might be used in the fitting procedure, such as
               the number of polinomyal degree. Variable will be
               initialized with the IDL execute command.

                    {\tt for i = 0, n\_elements(extra\_inputs)-1 do d = execute(extra\_inputs[i])}

               EXAMPLE:  

                {\tt extra\_inputs=['MOMENTS=2','DEGREE=-1','MDEGREE=3', 'BIAS=0.2', 
                'reddening=[0.1,0]', 'LAMBDA=exp(loglam\_gal)']}.

              Warning: The reddening (stars and/or gas) fit is
              optional, and it is performed by the Gandalf module. If
              the reddening fit is required, MDEGREE and DEGREE are
              used as the input value for the pPXF run, but
              automatically set to 0 and -1 respectively for the
              Gandalf execution. \\
%           
 star\_kin\_starting\_ guesses& [N x 4 fltarray]. The stellar kinematics starting guesses for V, 
                           $\sigma$, H3, and H4 for the N galaxy spectra to fit.
                           Default values are 0. for V, H3, and H4, and 50 km/sec for sigma. 
                           Starting guesses values are overrided by
                           the /use\_ previos\_ guesses keyword, if set.\\
%
 gas\_kin\_starting\_ guesses & [N x 2 fltarray]. The emission line kinematics starting guesses for V, 
                           $\sigma$, for the N galaxy spectra to fit.
                           Default values are 0 \kms\ for V, and 50 \kms\ for sigma. 
                           Starting guesses values are overrided by
                           the /use\_previos\_guesses keyword, if set.\\
%
emission\_line\_ file &[string]. It contains the name of the file with the information of the
                    emission lines to be fitted.
                    The format must be compatible with the module that performs the fit of the emission lines. 
                    If Gandalf or S-Gandalf are used, the input file must be an ascii file with the 
                    following 9 columns as in the following example (comments starts with ''\#''):
                      \medskip
 
                    {\tt 
                    \#  ID  CODE \  wav  \ action  line    Int	Vel     $\sigma$ mode

                    \#  \ \ \ \ \ \ \  \ \ [\AA]  \ \ f/i/m   dbl?

                    \#   0   HeII\ \ 3203.15  \   m \ \ \ l   \  \ 1.0	\   0 \	10 	t25

                    \#   1   NeV  3345.81 \    m  \ \ \ l    \ \ 1.0  \   0 \   10      t25

                    \ \  2   NeV  3425.81  \   m  \ \ \ l   \ \ 1.0   \  0  \  10      t25

                    \ \  3   OII  3726.03   \ m  \ \ \ l   \ \  1.0 \	   0 \	10 	t25 }

                     \medskip

                     \begin{itemize}
                       \item {\tt ID} Unique integer identifyer of the emission line.

                       \item {\tt CODE}. String. Name of the element. Special characters as ``[``, ``.'' are not permitted.

                       \item {\tt wav}. Float. Rest frame wavelength of the emission line to fit. Warning, the
                         name of the emission line in the final DAP output will be defined by the string {\tt CODE} +''\_''+ {\tt ROUND(wav)}.

                         \item {\tt action}. String. Possible values are ``f'' (fit), ``m'' (mask), and ``i''
                           (ignore). All lines are masked in the pPXF run, and then fit in the Gandalf run,
                           unless they are marked with ``i'', in this latter case they will be ignored.

                        \item {\tt line}. String Possible values are ``l'' (line) or ``dN'' (doublet). In this
                          case, N indicates the line ID to which the  doublet is linked, linked to the line with
                          ID=N. For example, if emission line with {\tt ID=4} has {\tt line=''d3''}, then the 
                          emission line with {\tt ID=3} must have {\tt line = ``l''}.

                        \item {\tt int}. Float. Relative intensity of the gas emission (positive) or absorption 
                               (negative) lines. It is set to 1 for lines ({\tt line = ``l''}). For doublets, ( {\tt line = ``dN''}) 
                               it indicates the ration elements of the doublets.
                        \item {\tt vel}. Float. Guess of the velocity offset with respect the galaxy systemic velocity (ignored in the DAP)
                        \item {\tt sigma}. Float. Guess of the Velocity dispersion (ignored in the DAP)
                        \item {\tt mode}. String. Possible values are ``f'' (fit) or ``tN'' (tied). If a line has {\tt mode = tN}, its kinematics is tied 
                                          to the line with {\tt ID=N}. In this case, the line with {\tt ID=N} must have {\tt mode = f}.
                         \end{itemize} \\
%
 range\_v\_star  &[2 elements array]. It specifies the boundaries for the stellar best fit velocity (in \kms). Default: starting\_guess $\pm$ 2000 \kms.\\
%;
 range\_s\_star  &[2 elements array]. It specifies the boundaries for the stellar best fit velocity dispersion (in \kms). Default: $21 < \sigma < 499$ \kms.\\
%;
 range\_v\_gas   &[2 elements array]. It specifies the boundaries for the emission line best fit velocity (in \kms) Default: starting\_guess $\pm$ 2000 \kms.\\
%;
 range\_s\_gas &[2 elements array]. It specifies the boundaries for the
             emission line best fit velocity dispersion (in \kms). Default:
             starting\_guess $\pm$ 2000\kms.\\
%; 
wavelength\_input & [QQ elements array]. If specified, it will be
             used to create wavelength\_output (in \AA), i.e. the
             wavelength vector to interpolate the final results on. if
             keyword /rest\_frame\_log is set, the vector is set to
             exp(loglam\_templates), and user inpiut will be
             overwritten. In this case QQ = MM.  Suggested
             entry: use the exp(loglam\_templates). \\
%
external\_library & String that specifies the path to the external {\tt FORTRAN} library, which contains the fortran versions of mdap\_bvls.pro. 
                  If not specified, or if the path is invalid, the default internal IDL mdap\_bvls code is used. \\
%
mask\_range  & If defined, it specifies the wavelength ranges to mask in the fit. It must contain an even number of entries, in
           angstrom. E.g. mask\_range=[$\lambda_0$, $\lambda_1$, $\lambda_2$, $\lambda_3$, ....$\lambda_{(2n-1)}$, $\lambda_{(2n)}$] will mask all the pixels where the 
             $\lambda_0 < exp({\rm loglam\_gal}) <\lambda_1$; $\lambda_2 < exp({\rm loglam\_gal}) <\lambda_3$;  $\lambda_{(2n-1)} < exp({\rm loglam\_gal}) <\lambda_{(2n)}$.\\
%
fwhm\_instr\_ kmsec\_matrix & 2xW elements vector. It defines the instrumental FWHM as function of wavelength. 
                               fwhm\_instr\_ kmsec\_matrix[0,*] specifies the wavelength (in angstrom, at which the instrumental FWHM is measured. 
                               fwhm\_instr\_ kmsec\_matrix[1,*] specifies values of the instrumental FWHM (in km/sec) measured at fwhm\_instr\_ kmsec\_matrix[0,*]. 
                               If undefined, a constant instrumental FWHM=122.45 km/sec is adopted. \\
\hline 
 {\bf OPTIONAL KEYWORDS}  &  \\
\hline
%;
 /use\_previos\_ guesses &  If set, the starting guesses for spectrum $i-$th
                       will be the best fit values from spectrum
                      $(i-1)-$th (i$>0$). Input starting guesses will be ignored.\\
%;
 /fix\_star\_kin  &        If set, the stellar kinematics are not
                       fitted. The return values is that of the starting guesses. \\
%
 /fix\_gas\_kin   &        If set, the emission-lines kinematics are not fitted. The return values is that of the starting guesses. \\
%
/quiet     &            If set, some information are not printed on screen.\\
%
/rest\_ frame\_log & If set, the output spectra (galaxy\_minus\_ ems\_fit\_model, best\_fit\_model, residuals, 
             best\_template, and best\_template\_LOSVD\_conv) are produced at rest-frame wavelength.\\
%%
\hline
%
 {\bf OUTPUTS} &  \\
\hline
%;
  stellar\_ kinematics   &  [N x 5 flt array].  It contains the best fit values of V, $\sigma$, h3, h4, and $\chi^2$/DOF for each of the N fitted input galaxy spectra. If /fix\_star\_kin is set, the array is not defined.\\
%;
  stellar\_ kinematics\_err &[N x 5 flt arrary].  It contains the errors to the best fit values of V, $\sigma$, h3, h4, and $\chi^2$/DOF for each of the N fitted input galaxy spectra.\\
%;
  stellar\_weights        &[N x NN dbl array]. It contains the weights of the NN templates for each of the N input galaxy spectra.\\
%;
  emission\_line\_ kinematics &[N x 2 flt array].  It contains the best fit values of V, $\sigma$ (emission lines) for each of the N fitted
                            input galaxy spectra. If /fix\_gas\_kin is set, the array is not defined.\\
%;
 emission\_line\_ kinematics\_err & [N x 2 flt array].  It contains the errors to the best fit values of V, $\sigma$ (emission lines) for each of the N fitted input galaxy spectra. If /fix\_gas\_kin is set, the array is not defined.\\
%;
 emission\_line\_ intens  &[N x T flt array].  It contains the intensities of the T fitted emission lines for each of the N input galaxy spectra. Values are corrected for reddening.\\
%
 emission\_line\_ intens\_err &  [N x T flt array].  Errors associated to emission\_line\_intens \\
%
 emission\_line\_ equivW      & [N x T flt array]. It contains the Equivalent widths of the T fitted emission lines for each of the N input galaxy spectra.                            Equivalent widths are computed by the ratio of emission\_line\_fluxes and the median value of the stellar spectrum within 5 and 10 $\sigma$ from the emission line. $\sigma$ is the emission line velocity dispersion. \\
%
 emission\_line\_ fluxes\_err &  [N x T flt array].  Errors associated to emission\_line\_fluxes \\
%
 emission\_line\_ equivW      & [N x T flt array]. It contains the Equivalent widths of the T fitted emission lines for each of the N input galaxy spectra.                            Equivalent widths are computed by the ratio of emission\_line\_fluxes and the median value of the stellar spectrum within 5 and 10 $\sigma$ from the emission line. $\sigma$ is the emission line velocity dispersion. \\
%
 emission\_line\_ equivW\_err  & [N x T flt array]. Errors associated to emission\_line\_equivW  . \\
%
 wavelength\_ output     & [QQ elements flt array]. It will contain the wavelength values over which the output spectra are sampled (in \AA). 
                        Default: it is set to wavelength\_input (if defined), or automatically computed with the smallest lambda/pixel step obtained from exp(loglam\_gal). 
                        It is set to exp(LOGLAM\_TEMPLATES) if the keyword /rest\_ frame\_log is set.\\
%
 best\_fit\_model      &[N x QQ flt array]. It will contain the best fit models for each of the input galaxy spectra (dereddended if 
                       MW\_extinction is not zero), sampled over wavelength\_output. It is in rest-frame if the keyword /rest\_frame\_log is set. \\
%
 galaxy\_minus\_ ems\_fit\_model &[N x QQ flt array]. It will contain the input galaxy spectra minus the emission lines best fit models (dereddended if MW\_extinction is not zero), 
                      sampled over wavelength\_output, for each of the N input spectra. It is in rest-frame if the keyword /rest\_frame\_log is set\\
%
 best\_template &[N x QQ flt array]. It will contain the best fitting template for each of the N input galaxy spectra sampled over wavelength\_output (rest frame wavelength). \\
%
 best\_template\_ LOSVD\_conv &[N x QQ flt array]. It will contain the best fitting template for each of the N input galaxy spectra convolved by best fitting 
LOSVD and sampled over wavelength\_output (rest frame wavelength).\\
%
reddening\_ output &[N x 2 elements array]. Best fit values for the reddening of stars (reddening\_ output[*,0]) and gas (reddening\_ output[*,1]) for all the N galaxy spectra. 
                                             If the reddening fit is not requred, the output value is set to 0. (reddening\_output[*,0:1] = [0,0]). 
                                             If only the reddening of the stars is fitted, the reddening of the gas is set to 0 (reddening\_output[*,1] = 0).\\
%
reddening\_ output\_err &[N x 2 elements array]. Errors associated to reddening\_ output. If not fitted, the error is automatically set to 99.\\
% 
 residuals &[N x QQ flt array]. It contains the difference between the observed galaxy spectra and the best\_fit\_model, 
                            sampled over wavelength\_output. It is in rest-frame if the keyword /rest\_frame\_log is set\\
%
\hline
{\bf  OPTIONAL OUTPUTS} &  \\
version & string specifying the module version. If requested, the module is not execute and only version flag is returned.\\
\hline
\hline
\end{longtable}
\end{center}
