\section{The configuration file}
\label{dap_sec:configuration}

The configuration file defines variables and parameters used in the
DAP. No empty lines should be present, commented lines are marked with
'\#'. The content of the configuration file is the following:

\begin{itemize}

 \item total\_filelist. A string indicating the full path to the file
   listing the galaxies to analyze and their physical parameters (See
   Section \ref{dap_sec:dap_inputs_outputs}).

 \item datacube\_root\_dir. Path indicating the location of the data
   to analyse. Files in datacube format must be stored in
   $<$datacube\_root\_dir$>$\textbackslash datacubes; files in RSS
   format must be stored in $<$datacube\_root\_dir$>$\textbackslash
   rss.
 
 \item output\_root\_dir. Path indicating where the results of the
   analysis should be stored. The directories
   $<$output\_root\_dir$>$\textbackslash resuts\_datacubes, and
   $<$output\_root\_dir$>$\textbackslash resuts\_rss must exist.

 \item w\_range\_for\_sn\_computation. Two elements array that
   specifies the wavelength range where to compute the signal to noise
   ratio. This array is passed to the main module
   mdap\_read\_datacube.pro, via the optional input {\tt lrange} (See
   Section \ref{dap_sec:mdap_read_datacube}). The suggested vaue for
   MANGA is to adopt the r-gunn FWHM bandpasse, centered at the
   effective wavelength (i.e.  $5560.00 < \lambda < 6942.00$
   \AA\ Fukugita et al. 1998). Leave it undefined to use the entire
   spectral range.


 \item w\_range\_for\_sn\_computation\_for\_gas. Two elements array
   that specifies the wavelength range where to compute the signal to
   noise ratio for emission line science. This range will be
   redshifted accordingly to the galaxy systemic velocity (from the
   starting guesses). If not specified, the default option will be
   adopted, i.e. use the same wavelenght range (with no redshift
   correction) for continuum and emission line science. One possible
   suggestion is to use the $6530 < \lambda <6600$ wavelength range,
   which embraces H$\alpha$ and NII emission lines.

  \item trim\_wav\_range\_spatial\_binning\_1. Two elements vector
    specifying the wavelength region to analyse in the first spatial
    binning (units: \AA). Binned galaxy spectra (of the first spatial
    binning) will be trimmed accordingly, templates will be trimmed
    over a sligtly larger wavelenth range.  Its value is passed to the
    main module mdap\_log\_rebin, during the rebinning and trimming of
    the spectra in the first spatial binning scheme through the
    optional input wave\_range (See Section
    \ref{dap_sec:mdap_spatial_binning}). If not defined, the entire
    available wavelength range will be used.


  \item trim\_wav\_range\_spatial\_binning\_2. As above, but for the
    second spatial binning scheme (3 spatial binning schemes are
    foreseen in the DAP, see Section \ref{dap_sec:block2}, plus a
    radial binning scheme \ref{dap_sec:block6}).

  \item trim\_wav\_range\_spatial\_binning\_3. As above, but for the
    third spatial binning scheme (3 spatial binning schemes are
    foreseen in the DAP, see Section \ref{dap_sec:block2}, plus a
    radial binning scheme \ref{dap_sec:block6}).

  \item trim\_wav\_range\_radial\_binning. As above, but for the
    radial spatial binning scheme (3 spatial binning schemes are
    foreseen in the DAP, see Section \ref{dap_sec:block2}, plus a
    radial binning scheme \ref{dap_sec:block6}).


  \item velscale. Float value indicating the velocity scale to adopt
    for the logarithmic rebinned spectra (units km/sec/pixel). This
    value is passed to mdap\_log\_rebin via the keyword
    input\_velscale (see Section \ref{dap_sec:mdap_log_rebin}), and to
    mdap\_spectral\_fitting via the velscale input variable (see
    Section \ref{dap_sec:mdap_spectral_fitting}). The default is to
    use the one automatically defined by the input galaxy
    specta. Suggested value: 30 km/sec/pixel.

  \item stellar\_library\_spatial\_binning\_1. String specifying the
    path. Warning: files are identified with the IDL function {\tt
      file\_search(stellar\_library\_spatial\_binning\_1)}. Therefore,
    be sure that this is enough to identify all and only the files
    that are needed. Warning: stellar\_library\_spatial\_binning\_1
    will be used also for the radial binning scheme. Files in the
    library must be fits files covering a wavelenght range
    preferentially larger than the MANGA ($3000< \lambda <
    10000$). They must have an uniform angstrom/pixel sampling. The
    following header keywords need to be present: CRVAL1 (value at
    reference pixel), CRPIX1 (reference pixel), and CDELT1 (dispersion
    in angstrom/pixel). stellar\_library\_spatial\_binning\_1 is
    required, and it will be passed to the interface mdap\_ log\_rebin
    and via the library input variable.

  \item stellar\_library\_spatial\_binning\_2 Same as
    stellar\_library\_spatial\_binning\_1, but for stars to be used
    with the spectra of the secon spatial binning.


  \item stellar\_library\_spatial\_binning\_3. Same as
    stellar\_library\_spatial\_binning\_1, but for stars to be used
    with the spectra of the third spatial binning.


  \item sn1\_rss. Float indicating the target signal-to-noise to adopt
    in the Voronoi binning scheme for RSS format data in the first
    spatial binning. This is mandatory and it will be passed to the
    interface mdap\_spatial\_binning via the min\_sn input variable
    (see Section \ref{dap_sec:mdap_spatial_binning}). Suggested entry
    = 15.

  \item sn2\_rss. Same of sn1\_rss, but for RSS format data in the
    first spatial binning. Suggested entry = 10.

  \item sn3\_rss. Same of sn1\_rss, but for RSS format data in the
    third spatial binning. Suggested entry = 5.

  \item sn1\_datacubes.  Same of sn1\_rss, but for datacube format
    data in the first spatial binning. Suggested entry = 40.

  \item sn2\_datacubes.  Same of sn1\_rss, but for datacube format
    data in the second spatial binning. Suggested entry = 25.

  \item sn3\_datacubes.  Same of sn1\_rss, but for datacube format
    data in the third spatial binning. Suggested entry = 15.

  \item sn\_thr\_tpl\_rss = 2. Threshold value for the S/N per
    angstrom each spectrum of RSS data format needs to have to be
    included in the analysis. S/N is computed over the wavelength
    range defined by the w\_range\_for\_sn\_computation variable (see
    above). Specra whose S/N are lower than this value will be
    discarded.Default = 0. To avoid any S/N threshold, set this
    variable to a very negative value (i.e. -100). Suggested value = 2

  \item sn\_thr\_str\_rss. Same as sn\_thr\_tpl\_rss, but for RSS
    format data in the second spatial binning scheme.  Suggested value
    = 2.

  \item sn\_thr\_ems\_rss. Same as sn\_thr\_tpl\_rss, but for RSS
    format data in the hird spatial binning scheme. Suggested value =
    2.

  \item sn\_thr\_tpl\_datacubes. Same as sn\_thr\_tpl\_rss, but for
    datacube format data in the first spatial binning scheme.
    Suggested value = 2.

  \item sn\_thr\_str\_datacubes.  Same as sn\_thr\_tpl\_rss, but for
    datacube format data in the second spatial binning scheme.
    Suggested value = 2.

  \item sn\_thr\_ems\_datacubes. Same as sn\_thr\_tpl\_rss, but for
    datacube format data in the third spatial binning scheme.
    Suggested value = 2.

  \item sn\_calibration\_rss. Calibration coefficients for RSS format
    files. Formula is specified in the mdap\_calibrat\_sn.pro
    function. If not defined no calibration will be used (seggested).
    sn\_calibration\_rss is passed to mdap\_spatial\_binnin.pro and
    mdap\_voronoi\_2d\_binning.pro via the sn\_calibration keyword


 \item sn\_calibration\_datacubes. Calibration coefficients for
   datacube format files. Formula is specified in the
   mdap\_calibrat\_sn.pro function, and it is given by the following
   expression:

   \[
    {\rm S/N_{REAL}} = \sum_{i=1,N} C_i \cdot \left( \frac{{\rm S/N_{ESTIMATED}}^{C_0}}{ \sqrt{{\rm N}_{\rm spax} }} \right)^{i-1} 
   \]

   where $C_i$ are the coefficients of sn\_calibration\_datacubes, and
   N$_{\rm spax}$ are the number of spaxels in that spatial bin.

   If sn\_calibration\_datacubes is not defined, the relation
   S/N$_{\rm REAL}$ = S/N$_{\rm ESTIMATED}$ is used. Suggested values
   for the MaNGA test run are: [1.1, 0.743865, 1.10317, --0.0106751,
     4.00892$\cdot$10$^{-5}$].

   Warning: Coefficient calibrations were not computed for optimally
   weighted binned spectra (i.e. do not use if weight\_for\_sn=1.)

   sn\_calibration\_datacubes is passed to mdap\_spatial\_binnin.pro
   and mdap\_voronoi\_2d\_ binning.pro via the sn\_calibration keyword
   (see Section \ref{dap_sec:mdap_spatial_binning}).

 \item weight\_for\_sn. If set to 0, the voronoi binning scheme uses
   the Modified Lyoid algorithm, and spectra belonging to the same
   spatial bin are added together with no weighting. If set to 1 the
   voronoi binning scheme uses an optimal weighting procedure (Signals
   and Errors is weighted by $S/N^2$), and spectra belonging to the
   same spatial bin are added with weights given by $S/N^2$. This
   value is passed to the interface mdap\_spatial\_binning.pro and the
   module mdap\_voronoi\_2d\_binning.pro via the keyword
   weight\_for\_sn (see Section
   \ref{dap_sec:mdap_spatial_binning}). If not defined, the default
   value (0) is used (no weights are applied). Warning: if the
   weighting is used, the suggested S/N calibration (see below) is not
   valid.

 \item user\_bin\_map\_spatial\_binning\_1. String specifying the fits
   file to be used to set the first spatial binning scheme. If
   provided, it will override the Voronoi binning scheme. Default: use
   the S/N to define the voronoi binning. Its value will be passed to
   the interface mdap\_spatial\_binning.pro via the keyword
   user\_bin\_map (see Section \ref{dap_sec:mdap_spatial_binning})

 \item user\_bin\_map\_spatial\_binnin\_2. Same as before, but for the
   second spatial binning scheme.

 \item user\_bin\_map\_spatial\_binnin\_3. Same as before, but for the
   third spatial binning scheme.

\item low\_radial\_bins\_user\_inputs. String array. Values of the
  lower boundaries of the radial binning scheme. Default: 6
  logarithmic binnings are created.

 \item upper\_radial\_bins\_user\_inputs. Values of the
  upper boundaries of the radial binning scheme. Default: 6
  logarithmic binnings are created.

 \item add\_default\_bins. If set, it adds the default 6 logarithmic
  radial binnings in addition to those defined by the user.

 \item emission\_line\_file\_spatial\_binnin\_1. String specifying the
   name of the file which defines the emission lines to be fitted in
   the first spatial binning and the fit set-up parameters (see
   Sections \ref{dap_sec:mdap_spectral_fitting} and
   \ref{dap_sec:mdap_gandalf}). Its value will be passed to the
   interface mdap\_spectral\_fitting.pro via the keyword
   emission\_line\_file (see Section
   \ref{dap_sec:mdap_spectral_fitting}).

 \item emission\_line\_file\_spatial\_binnin\_2. Same as before, but
   for the second spatial binning.

 \item emission\_line\_file\_spatial\_binnin\_3. Same as before, but
   for the third spatial binning.
    
 \item emission\_line\_file\_radial\_binning. Same as before, but for
   the radial binning.

 \item absorption\_line\_indices. String specifying the name of the
   file which defines the absorption line indices to measure. It must
   be an ascii file with 9 columns:

   \begin{itemize}

     \item Column 1. Integer. Unique ID number of the absorption line
       feature.

     \item Column 2.String. Unique name of the absorption line
       feature. This will define the name of the field in sctructure
       of the DAP results (i.e. the name must begin with a letter,
       special characters like comas or dots are not allowed).

     \item Columns 3-4. Float (units: \AA) Lower and upper values of
       the index passpand.

     \item Columns 5-6.  Float (units: \AA) Lower and upper values of
       the index blue pseudo-continuum.

     \item Columns 7-8.  Float (units: \AA) Lower and upper values of
       the index red pseudo-continuum.

     \item Column 9. String (accepted values are: ang or
       mag). Specifies the units (\AA\ or magnitudes) of the output.
 
    \end{itemize}
      
   absorption\_line\_indices must be defined, and it will be passed to
   the interface mdap\_ measure\_indices.pro and the utility
   mdap\_read\_indices\_definitions.pro via the
   absorption\_line\_indices input variable (see Sections
   \ref{dap_sec:mdap_measure_indice} and
   \ref{dap_sec:mdap_read_indices_definitions}).

   Indices will be measured only if their blue and red pesudo-continua
   bandpasses are included in the considered wavelength range. If not,
   their values are set to NaN, and their errors to 99 in the final
   output file.

  \item save\_intermediate\_steps. If set to 1 at the end of each
    block an idl session with all the current variables is
    saved. Default: 0 (suggested)

  \item remove\_null\_templates. If set to 1, the stellar templates
    that have null weight after the fiting of the first spatial bin
    will be rejected from the other fits. It is meaningful only if the
    libraries used for the other spatial binnings are the same of that
    used for the first spatial binning. Default: 0 Suggested value =
    1.

  \item external\_library. String indicating the location of fortran
    or C executables that will replace some of the IDL modules for
    speed purposes. If undefined, IDL routines will be used. The
    current DAP structure foresees the following two values, and
    mdap\_bvls.pro will be substituted by the mdap\_bvls\_external.pro
    routine:

    \begin{enumerate}

      \item external\_library= $<$mangadap$>$/external/F90\_32/ for 32
        bit machines;
 
      \item $<$mangadap$>$/external/F90\_64/ for 64 bit machines.

    \end{enumerate}

   Default: undefined (using internal routines). we suggest to define
   it and use external routines. The value of external\_library will
   be passed to the interface mdap\_spectral\_ fitting.pro via the
   keyword external\_library (if defined it will be used for all the
   spatial binning)

  \item spectra\_fittin\_parameters\_patial\_binning\_1. String array
    containing the additional inputs and keyword to be passed to the
    mdap\_spectral\_fitting.pro interface via the input keyword
    extra\_inputs (see Section \ref{dap_sec:mdap_spectral_fitting}) to
    be used to analyse the spectra of the first spatial binning.
    Suggested values: {\tt ['MOMENTS = 4', 'DEGREE = -1', 'MDEGREE =
        4']} Warning: variables will be defined via the IDL execute()
    function within the mdap\_spectral\_fitting.pro.

  \item spectra\_fittin\_parameters\_patial\_binning\_2. Same of
    before, but for the second spatial binning. Suggested value: {\tt
      ['MOMENTS = 4', 'DEGREE = -1', 'MDEGREE = 4']}

  \item spectra\_fittin\_parameters\_patial\_binning\_3.  Same of
    before, but for the third spatial binning. Suggested value: {\tt
      ['MOMENTS=4', 'MDEGREE=4', 'DEGREE=-1', 'reddening =
        [0.01,0.01]', 'LAMBDA = exp(loglam\_gal)']}

  \item spectra\_fittin\_parameters\_patial\_binning\_readial.  Same
    of before, but for the radial binning. Suggested value: {\tt
      ['MOMENTS=4', 'DEGREE=-1', 'mdegree=4', 'reddening = [0.01]',
        'LAMBDA = exp(loglam\_gal)']}

  \item instrumental\_fwhm\_file String indicating the name of the
    file with the wavelenght dependence of the instrumental FWHM. The
    file needs to be ascii with 4 columns.

  \begin{itemize}
     \item Column 1 : wavelength in angstrom, where the instrumental
       FWHM has been measured
     \item Column 2 : Resolving power.
     \item Column 3 : Instrumental FWHM in angstrom.
     \item Column 4 : Instrumental FWHM in km/sec.
  \end{itemize}

  Warning: Column 2 will not be used by the DAP, but it needs to be
  present to make the procedure running.

  If instrumental\_fwhm\_file is undefined, a constant instrumental
  FWHM of 2.54 \AA\ will be used.

\end{itemize}
