\section{{\tt mdap\_do\_log\_rebin.pro}}
\label{dap_sec:mdap_do_log_rebin}

This module is called by the mdap\_log\_rebin.pro interface and
performs the actual logarithmic resampling. It has been originally
written by M. Cappellari within the ppxf package. Here we report the
original description of the procedure written by M. Cappellari.

\medskip

 NAME: MDAP\_DO\_LOG\_REBIN

\medskip

 PURPOSE: Logarithmically rebin a spectrum, while rigorously
 conserving the flux (if the keyword $\backslash$flux is set).
 Basically the photons in the spectrum are simply ridistributed
 according to a new grid of pixels, with non-uniform size in the
 spectral direction.

 This routine makes the `standard' zero-order assumption that the
 spectrum is *constant* within each pixels. It is possible to perform
 log-rebinning by assuming the spectrum is represented by a piece-wise
 polynomial of higer degree, while still obtaining a uniquely defined
 linear problem, but this reduces to a deconvolution and amplifies
 noise.

 This same routine can be used to compute approximate errors of the
 log-rebinned spectrum. To do this type the command:
    \[
        {\rm MDAP\_DO\_LOG\_REBIN, lamRange, err}^2{\rm , err2New}
    \]
 and the desired errors will be given by SQRT(err2New).  NB: This
 rebinning of the error-spectrum is very *approximate* as it does not
 consider the correlation introduced by the rebinning!

The list of inputs/outputs of the mdap\_do\_log\_rebin.pro is given in
Table \ref{dap_tab:mdap_do_log_rebin}.

\begin{center}
\begin{longtable}{p{2.7cm}| p{11.1cm}}
\caption{Inputs and outputs parameters of mdap\_do\_log\_rebin.pro} \label{dap_tab:mdap_do_log_rebin} \\
\hline
\endfirsthead

\hline
\endhead

\hline
\endlastfoot

%\begin{tabular}{p{2.7cm}| p{2.5cm} |p{9cm}}
\hline
{\bf  INPUTS} &  \\
\hline
LAMRANGE & two elements vector containing the central wavelength
       of the first and last pixels in the spectrum, which is assumed
       to have constant wavelength scale! E.g. from the values in the
       standard FITS keywords: LAMRANGE = CRVAL1 + [0, CDELT1 * (NAXIS1-1)].
       It must be LAMRANGE[0] < LAMRANGE[1].\\
%
SPEC:& N elements array. The input spectrum. \\
%
\hline
{\bf  OUTPUTS} &  \\
\hline
SPECNEW & M elements array. The logarithmically rebinned spectrum. \\
%
 LOGLAM & log(lambda)M elements array. natural logarithm (ALOG) of the central 
       wavelength of each pixel. This is the log of the geometric 
       mean of the borders of each pixel.\\
%
\hline
{\bf  OPTIONAL KEYWORDS} &  \\
\hline
/FLUX: & Set this keyword to preserve total flux. In this case the 
       log rebinning changes the pixels flux in proportion to their 
       dLam so the following command will show large differences 
       beween the spectral shape before and after LOG\_REBIN:

       \medskip

        \ \   {\tt plot, exp(logLam), specNew}  

       \smallskip

        \ \   {\tt oplot, range(lamRange[0],lamRange[1],n\_elements(spec)), spec}

       \medskip

       By default, when this keyword is *not* set, the above two lines
       produce two spectra that almost perfectly overlap each
       other. Do not set /flux for MaNGA data.\\
%
OVERSAMPLE: & Oversampling can be done, not to loose spectral resolution, 
       especally for extended wavelength ranges and to avoid aliasing.
       Default: OVERSAMPLE=1, i.e. Same number of output pixels as input.\\
%
VELSCALE: &velocity scale in km/s per pixels. If this variable is
       not defined, then it will contain in output the velocity scale.
       If this variable is defined by the user it will be used
       to set the output number of pixels and wavelength scale.\\
\hline
\end{longtable}
\end{center}
