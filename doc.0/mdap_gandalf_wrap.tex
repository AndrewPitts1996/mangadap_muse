\section{{\tt mdap\_gandalf\_wrap.pro}}
\label{dap_sec:mdap_gandalf_wrap}

This main module fits an input galaxy spectrum with a series of stellar
templates and gas emission lines to derive the stellar and emission
lines kinematics, and the fluxes and equivalent width of the emission
lines.

The fit of the stellar kinematics is done using an implementation of
the pPXF routine (Cappellari \& Emsellem 2004, see Section
\ref{dap_sec:mdap_ppxf}), the fit of the emission line kinematics, the
weights of the stellar templates, (and reddening, if reuired) is done
using an implementation of the Gandalf routine (Sarzi et al. 2006, see
Section \ref{dap_sec:mdap_gandalf}).

The reddening of stars and gas (Balmer decrement) are fitted, if
required, using the Calzetti extinction law ( Calzetti et al. 2000,
see Section \ref{dap_sec:mdap_dust_calzetti}).

The steps performed by the {\tt mdap\_gandalf\_wrap.pro} main module
are:


\begin{itemize}

\item selection of regions to mask. These are:
   \begin{itemize} 

      \item the regions around emission lines; their location is
        computed using the velocity initial guess, and the width of
        the region to mask is set to 250 km/sec.

      \item The regions defined by the mask\_range input keyword, if
        provided.

   \end{itemize}

\item Run pPXF to measure the stellar kinematics. Additive and
  multiplicative polynomials are used in the fit, if specified in the
  user in the configuration file. The fit of the reddening is not
  performed. Note: The stellar kinematics is free to vary within the
  boundaries specified by the user (keywords range\_v\_star and
  range\_s\_star). This is an implementation to the original pPPXF
  procedure by Cappellari \& Emsellem (2004).

\item Remove the masks around the emission lines, but keeping the
  masks defined by the mask\_range input keyword (if defined).

\item Run Gandalf to measure the gas emission lines kinematics, and
  intensities. Additive polynomials are not used. Multiplicative
  polynomials are used (as in the previous pPXF run) only if the
  reddening is not fitted, otherwhise they are set to 0. Note: The gas
  kinematics is free to vary within the boundaries specified by the
  user (keywords range\_v\_gas and range\_s\_gas).  Also, the
  instrumental dispersion of the gas is assumed to vary with
  wavelength, therefore each emission line has its own value for the
  instrumental dispersion.  These are two implementations to the
  original Gandalf procedure by Sarzi et al. (2006).

\item Compute gas fluxes and equivalent widths (values are corrected
  for reddening, if reddening fit is required) from the emission line
  intensities.

\item Computation of the mean kinematics of the emission lines. The
  mean gas velocity and velocity dispersions are defined as the flux
  weighted average of the velocities and velocity dispersions of the
  detected individual emission lines.


\end{itemize}

Table \ref{dap_tab:mdap_gandalf_wrap} lists the inputs and outputs
parameters of the mdap\_gandalf\_wrap.pro module.


\begin{center}
\begin{longtable}{p{2.7cm}| p{11.1cm}}
\caption{Inputs and outputs parameters of
  mdap\_gandalf\_wrap.pro} \label{dap_tab:mdap_gandalf_wrap}
\\ \hline \endfirsthead

\hline
\endhead

\hline
\endlastfoot

%\begin{tabular}{p{2.7cm}| p{2.5cm} |p{9cm}}
\hline
{\bf  INPUTS} &  \\
\hline
templates & [MM x NN array].  It contains the NN stellar template
            spectra, logarithmically sampled at the same \kms/pixel as the
            galaxy spectra. Same units as galaxy, except an arbitrary
            multiplicative factor.\\
%
loglam\_templates & [MM dblarray]. It contains the log wavelength values where
            templates are sampled. It must have a constant log(angstrom) sampling.  \\
%
galaxy & [N elements array]. Galaxy spectrum, logarithically rebinned, to be fitted. \\
%
loglam\_gal & [N elements array]. $log(\lambda)$ values where the galaxy spectrum is defined. \\
%
noise & [N elements array]. Error vector associated to galaxy, defined over the loglam\_gal vector. \\
%
velscale & [float].  Defines the (uniform) sampling of the input spectra, in \kms/pixel.\\
% 
start\_ & 6 elements array containing the starting guesses:
            \ \ start\_[0] stellar veocity (km/sec)

            \ \ start\_[1] stellar velocity dispersion (km/sec)

             \ \ start\_[2] stellar h3 Gauss Hermite moment

            \ \ start\_[3] stellar h4 Gauss Hermite moment

            \ \ start\_[4] gas velocity (km/sec)

            \ \ start\_[5] gas velocity dispersion (km/sec). \\
%
\hline
{\bf  OPTIONAL INPUTS} &  \\
\hline
EMISSION\_ SETUP\_FILE & As in Table \ref{dap_tab:mdap_spectral_fitting}. \\
%
BIAS & As used Table  \ref{dap_tab:mdap_ppf}. \\
%
MDEGREE & Integer. Degree of multiplicative polynomials to be used in the pPXF fit and in the Gandalf fit (if reddening is not fitted). 
           Default: 0 (no multiplicative polynomials are used).\\
%
DEGREE &  Integer. Degree of multiplicative polynomials to be used in the pPXF only. 
           Default: -1 (no additive polynomials are used).\\
% 
reddening &  1 or 2 elements array. If specified in input, it triggers the
             fittind of the stellar reddening (1 element array) and the gas
             reddening (balmer decrement) (2 elements array). 
             In output it will contain the best fit reddening values. \\
%
 range\_v\_star  &[2 elements array]. It specifies the boundaries for the stellar best fit velocity (in \kms). Default: starting\_guess $\pm$ 2000 \kms.\\
%;
 range\_s\_star  &[2 elements array]. It specifies the boundaries for the stellar best fit velocity dispersion (in \kms). Default: $21 < \sigma < 499$ \kms.\\
%;
 range\_v\_gas   &[2 elements array]. It specifies the boundaries for the emission line best fit velocity (in \kms) Default: starting\_guess $\pm$ 2000 \kms.\\
%;
 range\_s\_gas &[2 elements array]. It specifies the boundaries for the
             emission line best fit velocity dispersion (in \kms). Default:
             starting\_guess $\pm$ 2000\kms.\\
%
mask\_range  & If defined, it specifies the wavelength ranges to mask in the fit. It must contain an even number of entries, in
           angstrom. E.g. mask\_range=[$\lambda_0$, $\lambda_1$, $\lambda_2$, $\lambda_3$, ....$\lambda_{(2n-1)}$, $\lambda_{(2n)}$] 
            will mask all the pixels where the 
             $\lambda_0 < exp({\rm loglam\_gal}) <\lambda_1$; $\lambda_2 < exp({\rm loglam\_gal}) <\lambda_3$; 
             $\lambda_{(2n-1)} < exp({\rm loglam\_gal}) <\lambda_{(2n)}$.\\
%
external\_library & String that specifies the path to the external {\tt FORTRAN} library, which contains the fortran versions of mdap\_bvls.pro. 
                  If not specified, or if the path is invalid, the default internal IDL mdap\_bvls code is used. \\
% 
INT\_DISP    &N elements array, containing the instrumental velocity dispersion (in km/sec) for all the N emission lines
               measured at their observed wavelengths (defined by the gas starting velocity guess). \\
\hline
{\bf KEYWORDS}  & \\
/FOR\_ERRORS    & If specified, it will trigger the computation of the emission lines error
                  (see Section \ref{dap_sec:mdap_gandalf}). Mandatory for the DAP workflow.\\
%
/fix\_star\_ kin & If set, the stellar kinematics will be fixed to the starting guesses values.\\
%
/fix\_gas\_ kin & If set, the gas kinematics will be fixed to the starting guesses values.\\
%
\hline
{\bf  OUTPUTS} &  \\
\hline
sol &  9 elements array containing the best fit kinematic parameters. 

       \ \ sol[0]:stellar velocity (km/sec).

       \ \ sol[1]:stellar velocity dispersion  (km/sec).

       \ \ sol[2]:stellar h3 Gauss-Hermite moment.

       \ \ sol[3]:stellar h4 Gauss-Hermite moment.

       \ \ sol[4]:stellar h5 Gauss-Hermite moment (not used).

       \ \ sol[5]:stellar h6 Gauss-Hermite moment (not used).

       \ \ sol[6]:$\chi^2$.

       \ \ sol[7]: mean flux weighted velocity of the emission lines (km/sec).

       \ \ sol[8]: mean flux weighted velocity dispersion (km/sec).  \\
%
gas\_intens &  N elements array, where N is the number of emission lines defined in EMISSION\_ SETUP\_FILE, 
                 whithin the wavelength range loglam\_gal. It specifies the intensity (corrected for reddening) 
                 of the emission lines. The intensities of lines in a multiplet are constrained by the flux ratio 
                 defined in the EMISSION\_ SETUP\_FILE (see Table \ref{dap_tab:mdap_spectral_fitting}). \\
%
gas\_fluxes & N elements array, where N is the number of emission lines defined in EMISSION\_ SETUP\_FILE, 
                 whithin the wavelength range loglam\_gal. It specifies the fluxes  (corrected for reddening) 
                 of the emission lines. The intensities of lines in a multiplet are constrained by the flux ratio 
                 defined in the EMISSION\_ SETUP\_FILE (see Table \ref{dap_tab:mdap_spectral_fitting}). \\
%
gas\_ew & N elements array, where N is the number of emission lines defined in EMISSION\_ SETUP\_FILE, 
                 whithin the wavelength range loglam\_gal. It specifies the equivalent widhts (corrected for reddening) 
                 of the emission lines. The intensities of lines in a multiplet are constrained by the flux ratio 
                 defined in the EMISSION\_ SETUP\_FILE (see Table \ref{dap_tab:mdap_spectral_fitting}). Equivalent widths 
                 are computed by comparing the emission line flux with the median flux of the stellar 
                 continuum in the spectral region  $\lambda_{i} - 10 \cdot FWHM_i < \lambda < \lambda_{i} - 5 \cdot FWHM_i $ and 
                 $\lambda_{i} +5 \cdot FWHM_i < \lambda < \lambda_{i} +10 \cdot FWHM_i $, where $\lambda_{i}$ is the central wavelength 
                  of the i-th emission line, and FWHM\_i is its measured FWHM (intrinsic plus instrumental). \\
%
gas\_intens\_err  & N elements array, error on gas\_intens. \\ 
%
gas\_fluxes\_err  & N elements array, error on gas\_fluxes. \\
%
gas\_ew\_err &N elements array, error on gas\_ew. \\
%
\hline
{\bf  OPTIONAL OUTPUTS} &  \\
\hline
bestfit   &   N elements array containing the best fit model (stars + gas), defined over the loglam\_gal vector. \\.
%
ERROR   & 8 elements array, containing the errors on the kinematic parameters.
       \ \ ERROR [0]: error on the stellar velocity (km/sec).

       \ \ ERROR [1]: error on the stellar velocity dispersion  (km/sec).

       \ \ ERROR [2]: error on the stellar h3 Gauss-Hermite moment.

       \ \ ERROR [3]: error on the stellar h4 Gauss-Hermite moment.

       \ \ ERROR [4]: error on the stellar h5 Gauss-Hermite moment (not used).

       \ \ ERROR [5]: error on the stellar h6 Gauss-Hermite moment (not used).

       \ \ ERROR [6]: error on the gas mean velocity (km/sec).

       \ \ ERROR [7]: error on the gas mean velocity dispersion (km/sec).  \\
%
reddening   & 1 or two elements array that contain the best fit reddening values 
            for star (1 element array) and for the gas (2 elements array). Input values will be overwritten.\\
%
err\_reddening & errors associated to reddening, if fitted.\\
%
fitted\_pixels &  array. Indices of the good pixels used in the gandalf fit.\\
%
status & bolean. If 0, the ppxf fit did not succeeded. The gandalf  fit is still carried on, and the stellar 
        kinematics are fixed to the starting guesses. If 1, the ppxf fit converged.\\
\hline
\hline
\end{longtable}
\end{center}
