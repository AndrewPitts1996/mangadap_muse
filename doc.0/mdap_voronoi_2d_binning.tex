\section{{\tt mdap\_voronoi\_2d\_binning.pro}}
\label{dap_sec:mdap_voronoi_2d_binning}

This procedure is taken from the Voronoi Binning procedure by
Cappellari \& Copin (2003).  It has been modified so it automatically
relaxes the minimal $S/N$ requirements to have at least 1 bin in the
field of view. Table \ref{dap_tab:mdap_voronoi_2d_binning} lists the
inputs and outputs required for this module.


\begin{center}
\begin{longtable}{p{2.7cm}| p{11.1cm}}
\caption{Inputs and outputs parameters of mdap\_voronoi\_2d\_binning} \label{dap_tab:mdap_voronoi_2d_binning} \\
\hline
\endfirsthead
\hline
\endhead
\hline
\endlastfoot
\hline
{\bf  INPUTS} &  \\
\hline
X &[flt array] Vector containing the X coordinate of the pixels to bin.
             Arbitrary units can be used (e.g. arcsec or pixels).
            In what follows the term ?pixel? refers to a given
            spatial element of the dataset (sometimes called ?spaxel? in
            the IFS community): it can be an actual pixel of a CCD
            image, or a spectrum position along the slit of a long-slit
            spectrograph or in the field of view of an IFS
            (e.g. a lenslet or a fiber).
            It is assumed here that pixels are arranged in a regular
            grid, so that the pixel size is a well defined quantity.
            The pixel grid however can contain holes (some pixels can be
            excluded from the binning) and can have an irregular boundary.
            See the above reference for an example and details.\\
%
Y  &[flt array]. Vector (same size as X) containing the Y coordinate
            of the pixels to bin.\\
%
SIGNAL  &[flt array]. Vector (same size as X) containing the signal
            associated with each pixel, having coordinates (X,Y).
            If the `pixels' are actually the apertures of an
            integral-field spectrograph, then the signal can be
            defined as the average flux in the spectral range under
            study, for each aperture.
            If pixels are the actual pixels of the CCD in a galaxy
            image, the signal will be simply the counts in each pixel.\\
%
NOISE  &[flt array]. Vector (same size as X) containing the corresponding
            noise (1 sigma error) associated with each pixel.\\
%
TARGETSN & [float]. The desired signal-to-noise ratio in the final
            2D-binned data. E.g. a S/N$\sim$50 per pixel may be a
            reasonable value to extract stellar kinematics
            information from galaxy spectra.\\
\hline
{\bf  OPTIONAL INPUT :}      &    \\
\hline
SN\_ CALIBRATION & vector. If provided, the estimated signal-to-noise
             ({\tt SN\_est}) is converted into the real signal-to-noise ({\tt
             SN\_real}) using the empirical calibration function defined in
            mdap\_calibrate\_sn.pro:
   \[
    {\rm S/N_{REAL}} = \sum_{i=1,N} C_i \cdot \left( \frac{{\rm S/N_{ESTIMATED}}^{C_0}}{ \sqrt{{\rm N}_{\rm spax} }} \right)^{i-1} 
   \]

where {\tt Nbin} is the number of spectra added in that spatial bin.\\
%
\hline
{\bf  INPUT KEYWORDS:}      &    \\
  /NO\_CVT&  Set this keyword to skip the Centroidal Voronoi Tessellation
           (CVT) step (vii) of the algorithm in Section 5.1 of
           Cappellari \& Copin (2003).
           This may be useful if the noise is strongly non Poissonian,
           the pixels are not optimally weighted, and the CVT step
           appears to introduces significant gradients in the S/N.
           A similar alternative consists of using the /WVT keyword below.\\
%
    /PLOT&   Set this keyword to produce a plot of the two-dimensional
           bins and of the corresponding S/N at the end of the
           computation.\\
%
   /QUIET&   by default the program shows the progress while accreting
           pixels and then while iterating the CVT. Set this keyword
           to avoid printing progess results.\\
%
     /WVT&   When this keyword is set, the routine bin2d\_cvt\_equal\_mass is
           modified as proposed by Diehl \& Statler (2006, MNRAS, 368, 497).
           In this case the final step of the algorithm, after the bin-accretion
           stage, is not a modified Centroidal Voronoi Tessellation, but it uses
           a Weighted Voronoi Tessellation.
           This may be useful if the noise is strongly non Poissonian,
           the pixels are not optimally weighted, and the CVT step
           appears to introduces significant gradients in the S/N.
           A similar alternative consists of using the /NO\_CVT keyword above.
           If you use the /WVT keyword you should also include a reference to
           `the WVT modification proposed by Diehl \& Statler (2006).'\\
%
/weight\_for\_sn & If set, the spectra in the same spatial bin will be
                 weighted by $S/N^2$ before being added. This is equivalent to adopt the following transformation:

                 \medskip

                  \ \ \ {\tt SIGNAL\_NEW = (SIGNAL\_OLD/NOISE\_OLD)}$^2$

                \medskip

                  \ \ \ {\tt NOISE\_NEW = SIGNAL\_OLD/NOISE\_OLD}   \\
                             %
\hline
{\bf  OUTPOUTS:}        &    \\
   BINNUMBER &[flt array]. Vector (same size as X) containing the bin number assigned
           to each input pixel. The index goes from zero to Nbins-1.
           This vector alone is enough to make *any* subsequent
           computation on the binned data. Everything else is optional!\\
%
     XBIN &[flt array].  Vector (size Nbins) of the X coordinates of the bin generators.
           These generators uniquely define the Voronoi tessellation.\\
%
     YBIN &[flt array].  Vector (size Nbins) of Y coordinates of the bin generators.\\
%
     XBAR &[flt array].  Vector (size Nbins) of X coordinates of the bins luminosity
           weighted centroids. Useful for plotting interpolated data.\\
%
     YBAR &[flt array].  Vector (size Nbins) of Y coordinates of the bins luminosity
           weighted centroids.\\
%
       SN &[flt array].  Vector (size Nbins) with the final SN of each bin.\\
%
  NPIXELS &[flt array].  Vector (size Nbins) with the number of pixels of each bin.\\
%
    SCALE &[flt array].  Vector (size Nbins) with the scale length of the Weighted
           Voronoi Tessellation, when the /WVT keyword is set.
           In that case SCALE is *needed* together with the coordinates
           XBIN and YBIN of the generators, to compute the tessellation
           (but one can also simply use the BINNUMBER vector).\\
\hline
\end{longtable}
\end{center}

