\subsection{{\tt mdap\_log\_rebin.pro}}
\label{dap_sec:mdap_log_rebin}

This interface controls the main module is responsable for the
logarithmic resampling of the input galaxy spexctra, errors, and the
stellar templates. It also broadens the input stellar spectra to match
the galaxy instrumental resolution. 

The list of input/output parameters defined in mdap\_log\_rebin.pro is
given in Table \ref{dap_tab:mdap_log_rebin}.



\begin{center}
\begin{longtable}{p{2.7cm}| p{11.1cm}}
\caption{Inputs and outputs parameters of mdap\_log\_rebin.pro} \label{dap_tab:mdap_log_rebin} \\
\hline
\endfirsthead
\hline
\endhead
\hline
\endlastfoot
\hline
{\bf  INPUTS} &  \\
\hline
spectra   &   [NxT dbl array].  The N galaxy spectra to resample.\\
%
errors    &   [NxT dbl array].  Errors associated to the N spectra.\\
%
wavelength & [T elements array].     Array with T elements, that specifies the wavelengths where galaxy spectra and errors are defined.\\
%
library    &  [string].          Directory containing the template spectra. Template spectra must be in fits format, defined over a 
                             linear wavelength with constant ang/pix step, and must contain all the header information needed to  
                             recover the wavelenght domain (i.e. CRPIX1, CDELT1, and CRVAL1).\\
fwhm\_diff & [T elements array].  Vector containing the FWHM($\lambda$) (in \AA) measured over the vector wavelength that the stellar spectra must be
                               convolved for. It correspond to the difference in quadrature between the instrumental FWHM of the observations, 
                                 and the instrumental FWHM of the templates. The convolution is done with the utility module 
                                mdap\_convo\_sigma.pro (see Section \ref{dap_sec:mdap_convol_sigma}). \\
%
\hline
{\bf  OPTIONAL INPUTS} & \\
\hline
input\_velscale & [flt].    Constant km/sec/pixel to be used when rebinning the input spectra. If not provided, the value will be
                             automatically set by the procedure.\\
%
wave\_range  & [2 elem array].  If specified, the galaxy spectra will be trimmed to this wavelength range (units: angstrom). Default: 
                             use the entire input wavelength range. Stellar spectra will be trimmed by 
                             wave\_range[0] - 250 ang and wave\_range[1] + 250 ang.\\

%
\hline
{\bf  OPTIONAL KEYWORDS}  & \\
\hline
/flux    &                  If set, flux conservation is applied to the log resampling. **Do not use** for template fitting. Suggestion: do not use.\\
%
/gal\_wavelength\_ log\_step &  Set this keyword if the input galaxy spectra are logarithmically sampled (i.e. wavelength has a logarithmic progression). 
                               Suggested for MaNGA data.\\
%
/quiet    &                    If set, message prompt is suppressed.\\
\hline
{\bf  OUTPUTS} &  \\
\hline
%
log\_spc  &  [N x TT dbl array]. The logarithmically resampled (ln-lambda) N galaxy spectra, over the wavelength range ln(wave\_range).\\
%
log\_err &   [N x TT dbl array]  The errors associated to the log\_spc. Errors are rebinned using the following formulas:


                              \ \ \   lrg=minmax(wavelength)

                              \ \ \   mdap\_do\_ log\_rebin, lrg, errors$^2$, log\_err2, loglam, velscale=velscale

                              \ \ \  log\_err = $\sqrt{{\rm log\_err2}}$

                             where mdap\_do\_log\_rebin.pro is the original procedure by M. Cappellari (see
                             Section \ref{dap_sec:mdap_do_log_rebin}).\\
%
log\_wav &   [TT dlb array].    The values of the ln-lambda over which log\_spc and log\_err are defined.\\
%
library\_log & [W x M dbl array]. The W stellar template spectra, logarithmically resampled.\\
%
log\_wav\_ library &[M dbl array]. The values of log-lambda over which the stellar templates are defined.\\
\hline
{\bf  OPTIONAL OUTPUTS} &  \\
\hline
version & string specifying the module version. If requested, the module is not execute and only version flag is returned.\\
\hline
\end{longtable}
\end{center}
